\documentclass[]{article}

%opening
\title{Método de Jacobi para resolver sistemas de ecuaciones lineales}
\author{Jonatan Enrique Badillo Tejeda y Javier Dolores Tolentino}

\begin{document}

\maketitle

\begin{abstract}
Este documento presenta una explicación del método de Jacobi para resolver sistemas de ecuaciones lineales, junto con dos implementaciones(secuencial y paralela) de este método en Java.
\end{abstract}

\section{Explicación del método de Jacobi}

El método de Jacobi es un algoritmo iterativo para resolver sistemas de ecuaciones lineales de la forma $Ax=b$ donde $A$ es una matriz de coeficientes, $x$ es el vector de incógnitas y $b$ es el vector de términos independientes.

El método de Jacobi descompone la matriz $A$ en una matriz diagonal $D$ y el resto $R$, de modo que $A=D+R$. Luego, se resuelve iterativamente la ecuación $Dx=b-Rx$ hasta que la solución converja.

Para cada iteración $k$ y para cada elemento $i$ del vector $x$, se calcula:

$x_i^{(k)} = \frac{1}{a_{ii}} \left( b_i - \sum_{j \neq i} a_{ij} x_j^{(k-1)} \right)$

Donde $a_{ij}$ son los elementos de la matriz $A$, $b_i$ son los elementos del vector $b$, y $x_j^{(k-1)}$ son los elementos del vector $x$ en la iteración anterior.

El algoritmo se detiene cuando la norma del error (la suma de las diferencias absolutas entre las soluciones actuales y las de la iteración anterior) es menor que un criterio de convergencia predefinido:

$\sum_{i=1}^{n} |x_i^{(k)} - x_i^{(k-1)}| < \epsilon$

Donde $\epsilon$ es el criterio de convergencia.

\section{Implementación en Java Secuencial}



El código en Java implementa el método de Jacobi para resolver sistemas de ecuaciones lineales de la forma $Ax = b$, donde $A$ es una matriz de coeficientes, $x$ es el vector de incógnitas, y $b$ es el vector de términos independientes.

En el código, primero se define la matriz $A$ y el vector $B$, que representan los coeficientes del sistema de ecuaciones y los términos independientes, respectivamente.

Se inicializan vectores para almacenar las soluciones iniciales y las soluciones de la iteración anterior. Además, se define un criterio de convergencia, representado por un pequeño valor de error.

Luego, se inicia un bucle while que continuará iterando hasta que se cumpla el criterio de convergencia. Dentro de este bucle, se realiza una serie de cálculos para actualizar las soluciones en cada iteración, utilizando el método de Jacobi. Se calcula la norma del error como la suma de las diferencias absolutas entre las soluciones actuales y las de la iteración anterior.

Si la norma del error es menor que el criterio de convergencia, el bucle se detiene y se imprimen las soluciones encontradas.



\begin{verbatim}
public class JacobiSecuencial {
    public static void main(String[] args) {
        // Define tu matriz A y el vector B aquí
        double[][] A = {
            {4, -1, 0, 0},
            {-1, 4, -1, 0},
            {0, -1, 4, -1},
            {0, 0, -1, 3}
        };
        
        double[] B = {15, 10, 10, 10};
        double[] X = new double[B.length]; // Vector de soluciones iniciales, inicializado a 0
        double[] lastX = new double[B.length]; // Vector para almacenar la solución de la iteración anterior
        double error = 1e-10; // Criterio de convergencia

        while (true) {
            for (int i = 0; i < A.length; i++) {
                double sum = B[i]; // Inicializa la suma con el elemento correspondiente del vector B
                for (int j = 0; j < A[i].length; j++) {
                    if (j != i) sum -= A[i][j] * lastX[j]; // Resta los productos Aij*Xj para j != i
                }
                X[i] = 1/A[i][i] * sum; // Divide por el coeficiente diagonal Aii
            }

            // Calcula la norma del error como la suma de las diferencias absolutas entre las soluciones actuales y las de la iteración anterior
            double errorNorm = 0;
            for (int i = 0; i < X.length; i++) {
                errorNorm += Math.abs(X[i] - lastX[i]);
                lastX[i] = X[i]; // Actualiza lastX con la solución actual para la próxima iteración
            }

            // Si la norma del error es menor que el criterio de convergencia, termina el bucle
            if (errorNorm < error) break;
        }

        // Imprime el resultado
        for (double x : X) System.out.println(x);
    }
}
\end{verbatim}

\section{Implementación del código Java paralelo}

Este código implementa el método de Jacobi en Java utilizando la biblioteca ForkJoinPool para paralelizar las operaciones. La matriz $A$ y el vector $b$ se definen al principio del código. El vector $X$ se inicializa a $0$ y se usa para almacenar las soluciones actuales, mientras que \texttt{lastX} se usa para almacenar las soluciones de la iteración anterior. El criterio de convergencia $\epsilon$ se establece en $1 \times 10^{-10}$.

El bucle \texttt{while} principal implementa las iteraciones del método de Jacobi. Para cada elemento $i$ del vector $x$, se calcula la suma de $b_i$ y los productos $A_{ij}X_j$ para $j \neq i$ (esta es la parte \texttt{sum -= A[i][j] * lastX[j]} en el código). Luego, se divide esta suma por el coeficiente diagonal $A_{ii}$ para obtener la solución actual $X_i$.

Después de calcular todas las soluciones actuales, se calcula la norma del error como la suma de las diferencias absolutas entre las soluciones actuales y las de la iteración anterior. Si la norma del error es menor que el criterio de convergencia, se termina el bucle.

Finalmente, se imprime el resultado, que es el vector de soluciones $x$.

La principal diferencia entre este código y el código secuencial anterior es que este código utiliza la biblioteca ForkJoinPool para paralelizar las operaciones. Esto puede resultar en un rendimiento significativamente mejorado para problemas grandes, ya que varias operaciones pueden realizarse simultáneamente en diferentes núcleos de la CPU.

\begin{verbatim}
    import java.util.concurrent.*;

public class JacobiParalelo {
    static final ForkJoinPool fjPool = new ForkJoinPool();

    static class JacobiTask extends RecursiveAction {
        final double[][] A;
        final double[] B;
        final double[] X;
        final double[] lastX;
        final int start;
        final int end;
        final double error;

        JacobiTask(double[][] A, double[] B, double[] X, double[] lastX, int start, int end, double error) {
            this.A = A;
            this.B = B;
            this.X = X;
            this.lastX = lastX;
            this.start = start;
            this.end = end;
            this.error = error;
        }

        protected void compute() {
            // Verifica si el tamaño del problema es pequeño
            if (end - start <= 500) { // Ajusta este valor según el tamaño de tu problema
                // Realiza el cálculo de Jacobi secuencialmente
                for (int i = start; i < end; i++) {
                    double sum = B[i];
                    for (int j = 0; j < A[i].length; j++) {
                        if (j != i) sum -= A[i][j] * lastX[j];
                    }
                    X[i] = 1/A[i][i] * sum;
                }
            } else {
                // Divide el trabajo y lo envía a tareas más pequeñas
                int mid = (start + end) / 2;
                invokeAll(new JacobiTask(A, B, X, lastX, start, mid, error),
                          new JacobiTask(A, B, X, lastX, mid, end, error));
            }
        }
    }

    public static void main(String[] args) {
        // Define tu matriz A y el vector B aquí
        double[][] A = {
            {4, -1, 0, 0},
            {-1, 4, -1, 0},
            {0, -1, 4, -1},
            {0, 0, -1, 3}
        };
        
        double[] B = {15, 10, 10, 10};
        double[] X = new double[B.length];
        double[] lastX = new double[B.length];
        double error = 1e-10;

        while (true) {
            // Invoca la tarea principal de Jacobi
            fjPool.invoke(new JacobiTask(A, B, X, lastX, 0, A.length, error));

            // Calcula la norma del error
            double errorNorm = 0;
            for (int i = 0; i < X.length; i++) {
                errorNorm += Math.abs(X[i] - lastX[i]);
                lastX[i] = X[i];
            }

            // Verifica el criterio de convergencia
            if (errorNorm < error) break;
        }

        // Imprime el resultado
        for (double x : X) System.out.println(x);
    }
}

    \end{verbatim}

    \section{Conclusión}

El método de Jacobi es un enfoque iterativo ampliamente utilizado para resolver sistemas de ecuaciones lineales. Aunque puede no ser la opción más eficiente en todos los casos, es relativamente simple de implementar y puede ser útil para sistemas de ecuaciones de tamaño moderado.

Las implementaciones secuenciales y paralelas del método de Jacobi presentadas en este documento demuestran cómo se puede aplicar este método en la práctica. La implementación secuencial es fácil de entender y puede ser adecuada para sistemas de ecuaciones pequeños o medianos. Por otro lado, la implementación paralela aprovecha el paralelismo de la CPU para mejorar el rendimiento en sistemas de ecuaciones más grandes, distribuyendo el trabajo entre varios núcleos de procesamiento.

En resumen, el método de Jacobi es una herramienta útil en el arsenal de técnicas para resolver sistemas de ecuaciones lineales, y su implementación secuencial y paralela en Java ofrece opciones flexibles para adaptarse a diferentes necesidades y recursos computacionales.

    
\end{document}

